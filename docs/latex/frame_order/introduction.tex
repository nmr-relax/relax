% Frame order introduction.
%%%%%%%%%%%%%%%%%%%%%%%%%%%

\section{Introduction of frame ordering}



% Tensors of frame ordering.
%~~~~~~~~~~~~~~~~~~~~~~~~~~~

\subsection{Tensors of frame ordering}

The frame order theory is defined as a bridging physics theory for rigid body motions based on the statistical mechanical ordering of reference frames.
It is implemented as a new analysis type in relax designed for the study of rigid body motions in molecules.
The theory aims to unify all rotational molecular physics techniques via a single statistical mechanical molecular dynamics (MD) model, by defining a series of rank-$2n$ frame order tensors.
These tensors encapsulate the maximum information content of a rotational molecular physics experiment, as well as what type of information is contained in the data.
To use the frame order theory, two steps are required:
\begin{enumerate}
\item The physics of the experiment should be decomposed into statistically mechanically averaged product of rotation matrix elements.
\item The MD model should be expressed in terms of a rotation matrix which modulates the motion of the rigid body, within the reference frame matching that of the experimental data.
\end{enumerate}




% RDC and PCS data.
%~~~~~~~~~~~~~~~~~~

\subsection{Ln$^{3+}$ aligned RDC and PCS data}

For the current implementation in relax, the frame order theory has been derived for molecules internally aligned using paramagnetic lanthanide ions.
The data required for the analysis includes both residual dipolar couplings (RDCs) and pseudocontact shifts (PCSs).
Both data sets are required as they are complementary, each carrying different dynamics information:
\begin{description}
\item[RDCs:]  This data is dominated by the amplitudes of the MD motions, and the orientation of the statical mechanical average structure.
\item[PCSs:]  These mainly provide information about the directions of the MD motions, and the orientation and position of the statical mechanical average structure.
\end{description}

For a successful analysis, the data must be of the highest quality.
In addition, multiple alignments are required, either using different lanthanide ions and/or a different attachment point for the lanthanide ion.
Note that the current implementation only handles alignment of one rigid body or domain.
