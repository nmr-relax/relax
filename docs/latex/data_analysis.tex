% Data analysis.
%%%%%%%%%%%%%%%%

\chapter{Data analysis}



% Introduction.
%%%%%%%%%%%%%%%

\section{Introduction}

This chapter aims to explain not only the steps involved in the data analysis of relaxation data but also how to use the program relax to achieve this.  Although a work in progress, it covers how to calculate the NOE, how to optimise and find the $\Rone$ and $\Rtwo$ relaxation rates, and how to implement model-free analysis.


% Calculating the NOE.
%%%%%%%%%%%%%%%%%%%%%%

\section{Calculating the NOE}
\index{NOE|textbf}

The calculation of NOE values is a straight forward and quick procedure which involves two components, the calculation of the value itself and the calculation of the errors.  To understand the steps involved, we will follow in detail the execution of a sample NOE calculation script.


% The sample script.
\subsubsection{The sample script}

\begin{exampleenv}
\# Script for calculating NOEs. \\
 \\
\# Create the run \\
name = `noe' \\
run.create(name, `noe') \\
 \\
\# Load the sequence from a PDB file. \\
pdb(name, `Ap4Aase\_new\_3.pdb', load\_seq=1) \\
 \\
\# Load the reference spectrum and saturated spectrum peak intensities. \\
noe.read(name, file=`ref.list', spectrum\_type=`ref') \\
noe.read(name, file=`sat.list', spectrum\_type=`sat') \\
 \\
\# Set the errors. \\
noe.error(name, error=3600, spectrum\_type=`ref') \\
noe.error(name, error=3000, spectrum\_type=`sat') \\
 \\
\# Individual residue errors. \\
noe.error(name, error=122000, spectrum\_type=`ref', res\_num=114) \\
noe.error(name, error=8500, spectrum\_type=`sat', res\_num=114) \\
 \\
\# Unselect unresolved residues. \\
unselect.read(name, file=`unresolved') \\
 \\
\# Calculate the NOEs. \\
calc(name) \\
 \\
\# Save the NOEs. \\
value.write(name, param=`noe', file=`noe.out', force=1) \\
 \\
\# Create grace files. \\
grace.write(name, y\_data\_type=`ref', file=`ref.agr', force=1) \\
grace.write(name, y\_data\_type=`sat', file=`sat.agr', force=1) \\
grace.write(name, y\_data\_type=`noe', file=`noe.agr', force=1) \\
 \\
\# View the grace files. \\
grace.view(file=`ref.agr') \\
grace.view(file=`sat.agr') \\
grace.view(file=`noe.agr') \\
 \\
\# Write the results. \\
results.write(name, file=`results', dir=None, force=1) \\
 \\
\# Save the program state. \\
state.save(`save', force=1) \\
\end{exampleenv}


% Initialisation of the run.
\subsubsection{Initialisation of the run}

Firstly to simplify referencing of the run name in the relevent functions, the name \texttt{`noe'} is assigned to to the object \texttt{name} by the command

\example{name = `noe'}

Therefore instead of typing \texttt{`noe'} each time the run needs to be referenced, \texttt{name} can be used instead.   The run is created by the command

\example{run.create(name, `noe')}

This user function will then create a run which is named \texttt{`noe'}, the second argument setting the run type to that of calculating the NOE.  Setting the run type is important so that the program knows which user functions are compatible with the run, for example the function \texttt{minimise()} is meaningless in this sample script as the NOE values are computed by direct calculation rather than through optimisation.


% Loading the data.
\subsubsection{Loading the data}

The first thing which need to be done prior to any residue specific command is to load the sequence.  In this case, the command

\example{pdb(name, `Ap4Aase\_new\_3.pdb', load\_seq=1)}
\index{PDB}

will extract the sequence from the PDB file `Ap4Aase\_new\_3.pdb'.  The first argument specifies the run into which the sequence will be loaded, the second specifies the file name, while the third causes the function to extract the sequence rather than just load the PDB into relax.  Although the PDB coordinates have been loaded into the program, the structure serves no purpose when calculating NOE values.

The next two commands

\begin{exampleenv}
noe.read(name, file=`ref.list', spectrum\_type=`ref') \\
noe.read(name, file=`sat.list', spectrum\_type=`sat') \\
\end{exampleenv}

load the peak heights of the reference and saturated NOE experiments (although the volume could be used instead).  The keyword argument \texttt{format} has not been specified, hence the default format of a Sparky \index{computer prgrams!Sparky} peak list (saved after typing \texttt{`lt'}) is assumed.  If the program XEasy \index{computer prgrams!XEasy} was used to analyse the spectra, the argument \texttt{format='xeasy'} is necessary.  The first column of the file should be the Sparky assignment string, while it is assumed that the 4$^\textrm{th}$ column contains either the peak height.


% Setting the errors.
\subsubsection{Setting the errors}

In this example, the errors where measured from the base plain noise.  The Sparky RMSD \index{RMSD} function was used to estimate the maximal noise levels across the spectrum in regions containing no peaks.  For the reference spectrum the RMSD was approximately 3600 while in the saturated spectrum the RMSD was 3000.  These errors are set by the commands

\begin{exampleenv}
noe.error(name, error=3600, spectrum\_type=`ref') \\
noe.error(name, error=3000, spectrum\_type=`sat') \\
\end{exampleenv}

For the residue G114, the noise levels are significantly increased compared to the rest of the protein as the peak is located close to the water signal.  The higher errors for this residue are specified by the commands

\begin{exampleenv}
noe.error(name, error=122000, spectrum\_type=`ref', res\_num=114) \\
noe.error(name, error=8500, spectrum\_type=`sat', res\_num=114) \\
\end{exampleenv}


% Unresolved residues.
\subsubsection{Unresolved residues}

As the peaks of certain residues overlaps to such an extent that the heights of both cannot be resolved, a simple text file was created called \texttt{unresolved} in which each line consists of a single residue number.  By using the command

\example{unselect.read(name, file=`unresolved')}

all residues in the file \texttt{unresolved} are excluded from the analysis.


% The NOE.
\subsubsection{The NOE}

At this point, the NOE can be calculated.  The user function

\example{calc(name)}

will calculate both the NOE and the errors.  The NOE value will be calculated using the formula
\begin{equation}
NOE = \frac{I_{sat}}{I_{ref}},
\end{equation}

\noindent where $I_{sat}$ is the intensity of the peak in the saturated spectrum while $I_{ref}$ is that of the reference spectrum.  The error is calculated by
\begin{equation}
\sigma_{NOE} = \sqrt{\frac{(\sigma_{sat} \cdot I_{ref})^2 + (\sigma_{ref} \cdot I_{sat})^2}{I_{ref}}},
\end{equation}

\noindent where $\sigma_{sat}$ and $\sigma_{ref}$ are the peak intensity errors in the saturated and reference spectra respectively.  To create a file of the NOEs, the command

\example{value.write(name, param=`noe', file=`noe.out', force=1)}

will create a file called \texttt{noe.out} with the NOE values and errors.  The force flag will cause any file with the same name to be overwritten.  An example of the format of \texttt{noe.out} is

{\footnotesize \begin{verbatim}
Num  Name  Value                         Error
1    GLY   None                          None
2    PRO   None                          None
3    LEU   None                          None
4    GLY   0.12479588727508535           0.020551827436105764
5    SER   0.42240815792914105           0.02016346825976852
6    MET   0.45281703194372114           0.026272719841642134
7    ASP   0.60727570079478255           0.032369427242382849
8    SER   0.63871921623680161           0.024695665815261791
9    PRO   None                          None
10   PRO   None                          None
11   GLU   None                          None
12   GLY   0.92927160307645906           0.059569089743604184
13   TYR   0.88832516377296256           0.044119641308479306
14   ARG   0.84945042565860407           0.060533543601110441
\end{verbatim}}


% Viewing the results.
\subsubsection{Viewing the results}

Any two dimensional data set can be plotted in relax in conjunction with the program Grace (http://plasma-gate.weizmann.ac.il/Grace/).  The program is also known as Xmgrace and was previously known as ACE/gr or Xmgr.  The highly flexible relax user function \texttt{grace.write} is capable of producing 2D plots of any x-y data sets.  The three commands

\begin{exampleenv}
grace.write(name, y\_data\_type=`ref', file=`ref.agr', force=1) \\
grace.write(name, y\_data\_type=`sat', file=`sat.agr', force=1) \\
grace.write(name, y\_data\_type=`noe', file=`noe.agr', force=1) \\
\end{exampleenv}

create three separate plots of the peak intensity of the reference and saturated spectra as well as the NOE.  The x-axis in all three defaults to the residue number.  As the x and y-axes can be any parameter, the command

\example{grace.write(name, x\_data\_type=`ref', y\_data\_type=`sat', file=`ref\_vs\_sat.agr', force=1)}

would create a plot of the reference verses the saturated intensity, with one point per residue.  Returning to the sample script, three Grace data files are created, \texttt{ref.agr}, \texttt{sat.agr}, and \texttt{noe.agr} and placed in the default directory \texttt{./grace}.  These can be visualised by opening the file within Grace, however relax will do that for you with the commands

\begin{exampleenv}
grace.view(file=`ref.agr') \\
grace.view(file=`sat.agr') \\
grace.view(file=`noe.agr') \\
\end{exampleenv}

An example of the output, after modifying the axes, is shown in figure~\ref{fig: NOE plot}.

\begin{figure}
\centerline{\includegraphics[width=0.8\textwidth, bb=0 0 792 612]{images/noe.eps.gz}}
\caption[NOE plot]{A Grace \index{computer programs!Grace|textbf} plot of the NOE value and error against the residue number.  An example of the output of the user function \texttt{grace.write()}.}\label{fig: NOE plot}
\end{figure}



% Relaxation curve fitting.
%%%%%%%%%%%%%%%%%%%%%%%%%%%

\newpage
\section{The $\Rone$ and $\Rtwo$ relaxation rates - relaxation curve fitting}
\index{relaxation curve fitting|textbf}



% Model-free analysis.
%%%%%%%%%%%%%%%%%%%%%%

\newpage
\section{Model-free analysis|textbf}
\index{model-free analysis}



% Reduced spectral density mapping.
%%%%%%%%%%%%%%%%%%%%%%%%%%%%%%%%%%%

\newpage
\section{Reduced spectral density mapping}
\index{reduced spectral density mapping|textbf}
